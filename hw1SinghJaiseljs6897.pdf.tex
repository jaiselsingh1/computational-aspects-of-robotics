\documentclass[11pt]{article}
\usepackage{amsmath}
\usepackage{amssymb}
\usepackage{geometry}
\usepackage{tikz}
\geometry{margin=1in}

\title{COMS W4733: Computational Aspects of Robotics\\Homework 1}
\author{Your Name \\ Your UNI}
\date{September 26, 2025}

\begin{document}
\maketitle

\section*{Problem 1: Homogeneous Transformations}

\subsection*{1. Convert to homogeneous coordinates (1 point)}
Converting $p_A^{\text{cart}} = \begin{bmatrix} 2 \\ 1 \\ 1 \end{bmatrix}$ to homogeneous coordinates:
$$p_A = \begin{bmatrix} p_A^{\text{cart}} \\ 1 \end{bmatrix} = \begin{bmatrix} 2 \\ 1 \\ 1 \\ 1 \end{bmatrix}$$

\subsection*{2. Construct $^A T_B$ (4 points)}
Frame $\{B\}$ is rotated 90° about the $z$-axis (counter-clockwise) relative to $\{A\}$:
$$R = R_z(90°) = \begin{bmatrix} 
\cos(90°) & -\sin(90°) & 0 \\
\sin(90°) & \cos(90°) & 0 \\
0 & 0 & 1
\end{bmatrix} = \begin{bmatrix}
0 & -1 & 0 \\
1 & 0 & 0 \\
0 & 0 & 1
\end{bmatrix}$$

With translation $t = \begin{bmatrix} 1 \\ -2 \\ 0 \end{bmatrix}$:
$$^A T_B = \begin{bmatrix}
R & t \\
0 & 0 & 0 & 1
\end{bmatrix} = \begin{bmatrix}
0 & -1 & 0 & 1 \\
1 & 0 & 0 & -2 \\
0 & 0 & 1 & 0 \\
0 & 0 & 0 & 1
\end{bmatrix}$$

\subsection*{3. Transform the point (3 points)}
To find the coordinates of $p$ in frame $\{B\}$:
$$^B T_A = (^A T_B)^{-1}$$

For a homogeneous transformation matrix:
$$^B T_A = \begin{bmatrix}
R^T & -R^T t \\
0 & 0 & 0 & 1
\end{bmatrix}$$

Computing:
$^B R_A = R^T = \begin{bmatrix}
0 & 1 & 0 \\
-1 & 0 & 0 \\
0 & 0 & 1
\end{bmatrix}$

$-R^T t = -\begin{bmatrix}
0 & 1 & 0 \\
-1 & 0 & 0 \\
0 & 0 & 1
\end{bmatrix} \begin{bmatrix} 1 \\ -2 \\ 0 \end{bmatrix} = -\begin{bmatrix} -2 \\ -1 \\ 0 \end{bmatrix} = \begin{bmatrix} 2 \\ 1 \\ 0 \end{bmatrix}$

Therefore:
$^B T_A = \begin{bmatrix}
0 & 1 & 0 & 2 \\
-1 & 0 & 0 & 1 \\
0 & 0 & 1 & 0 \\
0 & 0 & 0 & 1
\end{bmatrix}$

Computing $p_B = ^B T_A \cdot p_A$:
$p_B = \begin{bmatrix}
0 & 1 & 0 & 2 \\
-1 & 0 & 0 & 1 \\
0 & 0 & 1 & 0 \\
0 & 0 & 0 & 1
\end{bmatrix} \begin{bmatrix} 2 \\ 1 \\ 1 \\ 1 \end{bmatrix} = \begin{bmatrix} 0\cdot2 + 1\cdot1 + 0\cdot1 + 2\cdot1 \\ -1\cdot2 + 0\cdot1 + 0\cdot1 + 1\cdot1 \\ 0\cdot2 + 0\cdot1 + 1\cdot1 + 0\cdot1 \\ 1 \end{bmatrix} = \begin{bmatrix} 3 \\ -1 \\ 1 \\ 1 \end{bmatrix}$

Thus: $p_B^{\text{cart}} = \begin{bmatrix} 3 \\ -1 \\ 1 \end{bmatrix}$

\subsection*{4. Interpret the result (2 points)}
The 90° rotation about the $z$-axis swapped and negated the $x$ and $y$ components. The translation then shifted the point in the expected direction. The result is reasonable as the point moved from $(2,1,1)$ in frame $\{A\}$ to $(3,-3,1)$ in frame $\{B\}$.

\section*{Problem 2: Configuration Space and Workspace (Mobile Robot)}

\subsection*{1. C-space and DOF (2 points)}
\subsubsection*{(a) Configuration space}
$$\mathcal{Q} = \{(x, y, \theta) : x \in [0, 5], y \in [0, 4], \theta \in (-\pi, \pi]\} \subset \mathbb{R}^2 \times S^1$$

\subsubsection*{(b) Degrees of freedom}
The robot has 3 DOFs: 2 translational ($x, y$) and 1 rotational ($\theta$).

\subsection*{2. Position workspace of $P$ (5 points)}
\subsubsection*{(a) Ignoring the footprint}
$$\mathcal{W} = \{(x, y) : x \in [0, 5], y \in [0, 4]\} \subset \mathbb{R}^2$$

\subsubsection*{(b) With the footprint}
To avoid collision, point $P$ must stay at least $r_R = 0.35$ m away from walls:
$$\mathcal{W}_{\text{clear}} = \{(x, y) : x \in [0.35, 4.65], y \in [0.35, 3.65]\} \subset \mathbb{R}^2$$

\subsubsection*{(c) Quick check}
The point $(x, y) = (0.30, 0.30)$ is \textbf{not} reachable without collision since $(0.30, 0.30) \notin \mathcal{W}_{\text{clear}}$.

\subsection*{3. Workspace → C-space obstacles (2 points)}
\subsubsection*{(a) C-space obstacle}
Given obstacle $\mathcal{O} = \{(x, y) : \|(x, y) - (0.9, 0.3)\| \leq 0.10\}$

Using Minkowski sum with robot footprint:
$$\mathcal{Q}_{\text{obs}} = \{(x, y, \theta) : \|(x, y) - (0.9, 0.3)\| \leq 0.45\}$$

The obstacle does not depend on $\theta$ under the disc approximation.

\subsubsection*{(b) Configuration check}
For $q^* = (1.20, 0.40, \theta = 0.524)$:

Distance check:
$$d = \|(1.20, 0.40) - (0.9, 0.3)\| = \sqrt{(0.30)^2 + (0.10)^2} = \sqrt{0.10} \approx 0.316$$

Since $d = 0.316 < 0.45$, we have $q^* \in \mathcal{Q}_{\text{obs}}$.

\subsection*{4. Connectivity (1 point)}
$\mathcal{Q}_{\text{free}}$ is path-connected if there exists a continuous path in configuration space between any two points in $\mathcal{Q}_{\text{free}}$. This is essential for motion planning as it ensures a robot can move between any two valid configurations.

\section*{Problem 3: Forward Kinematics (2R Planar Arm)}

\subsection*{1. Geometric FK for position \& orientation (4 points)}
\subsubsection*{(a) Vector expression}
$$p_E = R(\theta_1) \begin{bmatrix} L_1 \\ 0 \end{bmatrix} + R(\theta_1 + \theta_2) \begin{bmatrix} L_2 \\ 0 \end{bmatrix}$$

Expanding:
$$p_E = \begin{bmatrix} L_1 \cos\theta_1 \\ L_1 \sin\theta_1 \end{bmatrix} + \begin{bmatrix} L_2 \cos(\theta_1 + \theta_2) \\ L_2 \sin(\theta_1 + \theta_2) \end{bmatrix}$$

\subsubsection*{(b) Scalar formulas}
$$x(\theta_1, \theta_2) = L_1 \cos\theta_1 + L_2 \cos(\theta_1 + \theta_2)$$
$$y(\theta_1, \theta_2) = L_1 \sin\theta_1 + L_2 \sin(\theta_1 + \theta_2)$$

\subsubsection*{(c) End-effector orientation}
$$\phi(\theta_1, \theta_2) = \theta_1 + \theta_2$$

The end-effector orientation is the sum of both joint angles as each contributes additively to the total rotation.

\subsection*{2. Pose in SE(2) (3 points)}
\subsubsection*{(a) Homogeneous transform}
$$^0 T_E = \begin{bmatrix}
\cos(\theta_1 + \theta_2) & -\sin(\theta_1 + \theta_2) & L_1\cos\theta_1 + L_2\cos(\theta_1 + \theta_2) \\
\sin(\theta_1 + \theta_2) & \cos(\theta_1 + \theta_2) & L_1\sin\theta_1 + L_2\sin(\theta_1 + \theta_2) \\
0 & 0 & 1
\end{bmatrix}$$

\subsubsection*{(b) Product of elementary transforms}
$$^0 T_E = ^0 T_1 \cdot ^1 T_E$$

where:
$$^0 T_1 = \begin{bmatrix}
\cos\theta_1 & -\sin\theta_1 & L_1\cos\theta_1 \\
\sin\theta_1 & \cos\theta_1 & L_1\sin\theta_1 \\
0 & 0 & 1
\end{bmatrix}$$

$$^1 T_E = \begin{bmatrix}
\cos\theta_2 & -\sin\theta_2 & L_2\cos\theta_2 \\
\sin\theta_2 & \cos\theta_2 & L_2\sin\theta_2 \\
0 & 0 & 1
\end{bmatrix}$$

\subsection*{3. Numeric evaluation (2 points)}
For $\theta_1 = 30° = \pi/6$ rad and $\theta_2 = 60° = \pi/3$ rad:
\begin{align}
\phi &= \theta_1 + \theta_2 = \pi/6 + \pi/3 = \pi/2 = 1.571 \text{ rad} \\
x &= 1.0\cos(\pi/6) + 0.8\cos(\pi/2) = 0.866 + 0 = 0.866 \text{ m} \\
y &= 1.0\sin(\pi/6) + 0.8\sin(\pi/2) = 0.5 + 0.8 = 1.300 \text{ m}
\end{align}

$$^0 T_E = \begin{bmatrix}
0 & -1 & 0.866 \\
1 & 0 & 1.300 \\
0 & 0 & 1
\end{bmatrix}$$

\subsection*{4. Tool offset (gripper) (1 point)}
Gripper transform:
$$^E T_G = \begin{bmatrix}
1 & 0 & 0.1 \\
0 & 1 & 0 \\
0 & 0 & 1
\end{bmatrix}$$

For the gripper position:
\begin{align}
x_G &= x_E + d_g\cos\phi = 0.866 + 0.1\cos(\pi/2) = 0.866 \text{ m} \\
y_G &= y_E + d_g\sin\phi = 1.300 + 0.1\sin(\pi/2) = 1.400 \text{ m}
\end{align}

\section*{Problem 4: Inverse Kinematics (2R Planar Arm)}

\subsection*{1. Reachability condition (2 points)}
For a point $(x, y)$ to be reachable:
$$|L_1 - L_2| \leq r \leq L_1 + L_2$$

where $r = \sqrt{x^2 + y^2}$. This represents the annulus between the inner circle (arm folded) and outer circle (arm extended).

\subsection*{2. Elbow angle $\theta_2$ (3 points)}
Using the law of cosines:
$$\cos\theta_2 = \frac{x^2 + y^2 - L_1^2 - L_2^2}{2L_1L_2}$$

Therefore:
$$\theta_2 = \pm \arccos\left(\frac{x^2 + y^2 - L_1^2 - L_2^2}{2L_1L_2}\right)$$

Two branches:
- Elbow-up: $\theta_2 > 0$ (positive branch)
- Elbow-down: $\theta_2 < 0$ (negative branch)

\subsection*{3. Shoulder angle $\theta_1$ (3 points)}
$$\theta_1 = \text{atan2}(y, x) - \text{atan2}(L_2\sin\theta_2, L_1 + L_2\cos\theta_2)$$

This gives two $\theta_1$ values corresponding to the two $\theta_2$ branches.

\pi/4, 3\pi/4]$:
- Elbow-up: $\theta_1 = -0.414$ ✓, $\theta_2 = 0.927$ ✗ (exceeds $3\pi/4 \approx 2.356$)
- Elbow-down: $\theta_1 = 1.058$ ✓, $\theta_2 = -0.927$ ✓

Only the elbow-down configuration satisfies the joint limits.

Forward check verification confirms the solution is correct.

\end{document}