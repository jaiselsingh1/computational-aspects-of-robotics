\documentclass[11pt]{article}
\usepackage{amsmath}
\usepackage{amssymb}
\usepackage{geometry}
\geometry{margin=1in}

\title{COMS W4733: Computational Aspects of Robotics\\Homework 1}
\author{Your Name \\ Your UNI}
\date{September 26, 2025}

\begin{document}
\maketitle

\section*{Problem 1: Homogeneous Transformations}

\subsection*{1. Convert to homogeneous coordinates (1 point)}
Convert $p_A^{\text{cart}} = \begin{bmatrix} 2 \\ 1 \\ 1 \end{bmatrix}$ to homogeneous:
\[
p_A = \begin{bmatrix} p_A^{\text{cart}} \\ 1 \end{bmatrix}
= \begin{bmatrix} 2 \\ 1 \\ 1 \\ 1 \end{bmatrix}.
\]

\subsection*{2. Construct $^{A}\!T_{B}$ (4 points)}
Rotation $90^\circ$ about $+z$:
\[
R = R_z(90^\circ) =
\begin{bmatrix}
0 & -1 & 0\\
1 & \phantom{-}0 & 0\\
0 & \phantom{-}0 & 1
\end{bmatrix},
\qquad
t = \begin{bmatrix} 1 \\ -2 \\ 0 \end{bmatrix}.
\]
Therefore
\[
^{A}\!T_{B}=\begin{bmatrix}R&t\\0~0~0&1\end{bmatrix}
=
\begin{bmatrix}
0 & -1 & 0 & 1\\
1 & \phantom{-}0 & 0 & -2\\
0 & \phantom{-}0 & 1 & 0\\
0 & \phantom{-}0 & 0 & 1
\end{bmatrix}.
\]

\subsection*{3. Transform the point (3 points)}
Use $(^{A}\!T_{B})^{-1}=\begin{bmatrix}R^\top&-R^\top t\\ 0&1\end{bmatrix}$:
\[
^{B}\!T_{A}=
\begin{bmatrix}
0 & 1 & 0 & 2\\
-1 & 0 & 0 & 1\\
0 & 0 & 1 & 0\\
0 & 0 & 0 & 1
\end{bmatrix},\qquad
p_B = {}^{B}\!T_{A}\,p_A =
\begin{bmatrix} 3\\ -1\\ 1\\ 1\end{bmatrix}.
\]
So $p_B^{\text{cart}}=(3,-1,1)^\top$.

\subsection*{4. Interpret the result (2 points)}
The $90^\circ$ rotation swaps $(x,y)\mapsto(-y,x)$; the translation adds $(+1,-2,0)$, yielding $(3,-1,1)$ as expected.

\section*{Problem 2: Configuration Space and Workspace (Mobile Robot)}

\subsection*{1. C-space and DOF (2 points)}
\noindent
(a) $q=(x,y,\theta)\in\mathbb{R}^2\times S^1$ with
\[
\mathcal{Q}=[0,5]\times[0,4]\times(-\pi,\pi].
\]
(b) $\mathrm{DOF}=3$ (two translational, one rotational).

\subsection*{2. Position workspace of $P$ (5 points)}
(a) Ignoring footprint:
\[
\mathcal{W}=[0,5]\times[0,4].
\]
(b) With disc footprint $r_R=0.35$:
\[
\mathcal{W}_{\text{clear}}=\{(x,y): 0.35\le x\le 4.65,\ 0.35\le y\le 3.65\}.
\]
(c) $(0.30,0.30)\notin\mathcal{W}_{\text{clear}}$ (collision).

\subsection*{3. Workspace $\to$ C-space obstacles (2 points)}
Obstacle
\(
\mathcal{O}=\{(x,y):\|(x,y)-(0.9,0.3)\|\le 0.10\}.
\)
Grow by $r_R$:
\[
\|(x,y)-(0.9,0.3)\| \le 0.10+0.35=0.45
\iff
(x-0.9)^2+(y-0.3)^2 \le 0.45^2=0.2025.
\]
Thus
\(
\mathcal{Q}_{\text{obs}}=\{(x,y,\theta): (x-0.9)^2+(y-0.3)^2\le 0.2025\}.
\)
For $q^\star=(1.20,0.40,0.524)$:
$(1.20-0.9)^2+(0.40-0.3)^2=0.10<0.2025\Rightarrow q^\star\in\mathcal{Q}_{\text{obs}}$.

\subsection*{4. Connectivity (1 point)}
$\mathcal{Q}_{\text{free}}$ is \emph{path-connected} if any two configurations in it are connected by a continuous collision-free path.

\section*{Problem 3: Forward Kinematics (2R Planar Arm)}

\subsection*{1. Geometric FK for position \& orientation (4 points)}
(a) Vector form:
\[
p_E =
\underbrace{R(\theta_1)\!\begin{bmatrix}L_1\\0\end{bmatrix}}_{\text{Link 1}}
+
\underbrace{R(\theta_1+\theta_2)\!\begin{bmatrix}L_2\\0\end{bmatrix}}_{\text{Link 2}},
\quad
R(\alpha)=\begin{bmatrix}\cos\alpha&-\sin\alpha\\ \sin\alpha&\cos\alpha\end{bmatrix}.
\]
(b) Scalars:
\[
x=L_1\cos\theta_1+L_2\cos(\theta_1+\theta_2),\quad
y=L_1\sin\theta_1+L_2\sin(\theta_1+\theta_2).
\]
(c) Orientation: $\phi=\theta_1+\theta_2$ (since $\theta_2$ is relative).

\subsection*{2. Pose in $\text{SE}(2)$ (3 points)}
\[
{}^{0}\!T_{E}=
\begin{bmatrix}
\cos\phi&-\sin\phi&x\\
\sin\phi& \cos\phi&y\\
0&0&1
\end{bmatrix}
=
\underbrace{R_z(\theta_1)T_x(L_1)}_{^{0}\!T_1}\,
\underbrace{R_z(\theta_2)T_x(L_2)}_{^{1}\!T_E}.
\]

\subsection*{3. Numeric evaluation (2 points)}
For $\theta_1=30^\circ=\pi/6$ and $\theta_2=60^\circ=\pi/3$ with $L_1=1.0$, $L_2=0.8$:
\[
\phi=1.571,\quad x=0.866,\quad y=1.300,\qquad
{}^{0}\!T_E=\begin{bmatrix}0&-1&0.866\\ 1&0&1.300\\ 0&0&1\end{bmatrix}.
\]

\subsection*{4. Tool offset (gripper) (1 point)}
With $d_g=0.10$ along $x_E$:
\[
{}^{E}\!T_G=T_x(d_g),\quad
{}^{0}\!T_G={}^{0}\!T_E{}^{E}\!T_G=
\begin{bmatrix}
\cos\phi&-\sin\phi&x+d_g\cos\phi\\
\sin\phi& \cos\phi&y+d_g\sin\phi\\
0&0&1
\end{bmatrix}.
\]
Numerically $(x_G,y_G)=(0.866,1.400)$.

\section*{Problem 4: Inverse Kinematics (2R Planar Arm)}

\subsection*{1. Reachability condition (2 points)}
Let $r=\sqrt{x^2+y^2}$. The point $(x,y)$ is reachable iff
\[
\boxed{|L_1-L_2|\le r\le L_1+L_2},
\]
i.e., the target lies in the annulus between the inner (arm folded) and outer (arm stretched) circles.

\subsection*{2. Elbow angle $\theta_2$ (3 points)}
Law of cosines on triangle $(L_1,L_2,r)$ with elbow interior angle $\pi-\theta_2$:
\[
\cos\theta_2=\frac{r^2-L_1^2-L_2^2}{2L_1L_2}=:c_2,\qquad
s_2=\pm\sqrt{1-c_2^2},\qquad
\boxed{\ \theta_2=\mathrm{atan2}(s_2,c_2)\ }.
\]
Two branches: \emph{elbow-up} ($s_2>0$) and \emph{elbow-down} ($s_2<0$).

\subsection*{3. Shoulder angle $\theta_1$ (3 points)}
Let $\alpha=\mathrm{atan2}(y,x)$ and $\beta=\mathrm{atan2}\!\big(L_2 s_2,\ L_1+L_2 c_2\big)$. Then
\[
\boxed{\ \theta_1=\alpha-\beta
= \mathrm{atan2}(y,x)-\mathrm{atan2}\!\big(L_2 s_2,\ L_1+L_2 c_2\big)\ }.
\]
This yields one $\theta_1$ for each choice of $\mathrm{sign}(s_2)$.

\subsection*{4. Numeric test \& joint limits (2 points)}
Target $x^\star=1.20$, $y^\star=0.40$; link lengths $L_1=1.0$, $L_2=0.8$; limits
\[
\theta_1\in[-\pi,\pi),\qquad \theta_2\in\Big[-\frac{3\pi}{4},\,\frac{3\pi}{4}\Big]=[-2.356,\,2.356].
\]

\paragraph{Compute $r$, $c_2$, $s_2$.}
\[
r=\sqrt{1.20^2+0.40^2}=1.265,\quad
c_2=\frac{r^2-L_1^2-L_2^2}{2L_1L_2}=\frac{1.265^2-1.0^2-0.8^2}{2(1)(0.8)}=-0.025.
\]
\[
s_2=\pm\sqrt{1-c_2^2}=\pm 0.999687.
\]

\paragraph{Angles for the two branches.}
\[
\alpha=\mathrm{atan2}(0.40,1.20)=0.322,\qquad
\beta=\mathrm{atan2}(L_2 s_2,\ L_1+L_2 c_2)=\mathrm{atan2}(0.8\,s_2,\ 0.980).
\]
\[
\begin{array}{lcl}
\text{Elbow-up }(s_2>0): & \theta_2=+1.596, & \theta_1=\alpha-\beta=-0.363,\\[2pt]
\text{Elbow-down }(s_2<0): & \theta_2=-1.596, & \theta_1=\alpha-\beta=+1.006.
\end{array}
\]
(Values rounded to 3 decimals.)

\paragraph{Joint-limit check.}
Both sets lie within $\theta_1\in[-\pi,\pi)$ and $\theta_2\in[-2.356,2.356]$:
\[
\boxed{(-0.363,\ +1.596)}\ \text{(elbow-up)}\quad\text{and}\quad
\boxed{(+1.006,\ -1.596)}\ \text{(elbow-down)}\ \ \text{are valid.}
\]

\paragraph{Forward check (tolerance $10^{-3}$).}
Using the FK from Problem~3,
\[
\hat x=L_1\cos\theta_1+L_2\cos(\theta_1+\theta_2),\qquad
\hat y=L_1\sin\theta_1+L_2\sin(\theta_1+\theta_2),
\]
both branches return $(\hat x,\hat y)=(1.200,\,0.400)$ (error $<10^{-12}$), hence the target is met.

\end{document}
